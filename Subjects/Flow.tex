\subsection{Max Flow \& Min Cut}
\subsection{Flow Algorithms}
\begin{table}[H]
    \centering
    \begin{tabular}{c|c|c|c}
        Name & Running Time & Explanation & Notes \\
        \hline
        Ford-Fulkerson & $\BigO(|f^{*}|m)$ & 
            \makecell{$|f^{*}|$ = max flow value \\ $m$ = \# edges} & 
            \makecell{Pseudopolynomial; \\ can fail to terminate} \\
        \hline 
        Scaling & $\BigO(m^2 \log C)$ & 
            $C$ = max edge capacity & 
            \makecell{Weakly polynomial; \\ assumes integer capacities} \\
        \hline
        Edmonds-Karp & $\BigO(m^2 n)$ & 
            \makecell{$n$ = \# vertices \\ $m$ = \# edges} & 
            \makecell{Uses BFS; \\ strongly polynomial} \\
    \end{tabular}
\end{table}
\subsection{Running the algorithms}
When running the algorithms by hand, just use a basic approach—it's manageable 
when the graph is reasonably simple. For min-cut, find the saturated edges 
(at capacity) along the residual graph boundary between reachable and 
unreachable vertices from the source.

\subsection{Flow Problem Recipe}
\subsubsection{Build the Graph}
\begin{itemize}
    \item Vertices (source + sink)
    \item Edges (capacities + direction)
\end{itemize}
\subsubsection{Describe the Algorithm}
\begin{itemize}
    \item Find max flow ($m$) and INSERT ALGORITHM
    \item Return an answer
\end{itemize}
\subsubsection{Analyse Runtime}
\begin{itemize}
    \item Build Parameters from question
    \item Remember the time it takes to build the graph
\end{itemize}
